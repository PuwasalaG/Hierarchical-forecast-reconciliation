\documentclass[a4paper,11pt]{article}
\newif\ifblind
\blindfalse %Change this to \blindtrue or \blindfalse depending on whether paper should be blind.
\usepackage{dcolumn}
\newcolumntype{d}[1]{D{.}{.}{-1}}

\usepackage{xcolor,colortbl}
\usepackage{amsmath}
\usepackage{amsfonts}
\usepackage{amssymb}
\usepackage{caption} %captionsf
\usepackage{subcaption} %subcaptions
\usepackage{bbm} % more maths symbols
\usepackage{enumitem} % a b c itemize
\usepackage{pdfpages} % pdf as graphic
\makeatletter % so @ can be used
\setlength{\@fptop}{0pt} % Start graphic at top
\makeatother % turn off @
\usepackage{multirow} % Multirow in table
\usepackage{booktabs,caption,fixltx2e} % table caption with notes
\usepackage[flushleft]{threeparttable} % table caption with notes
\usepackage{setspace} % This is used in the title page
\usepackage{graphicx} % This is used to load the crest in the title page
\usepackage{geometry}
\usepackage{bm}
\usepackage{sectsty}
\usepackage{amssymb} % leqslant
\setcounter{secnumdepth}{3}
\setcounter{tocdepth}{3}

\begin{document}
	
	%\section{GENERAL COMMENTS TO THE REVIEWERS AND EDITOR}
	
	\section*{Detailed Response to Referee 1: Summary}
	
	\textit{The authors provide a geometric interpretation for reconciliation of hierarchical forecasts. They show why and how reconciliation via projection is guaranteed to improve squared forecast errors. They explore a couple of different ways for dealing with biased base forecasts in an application to Australian tourism flows. Overall the paper is well written and the geometric interpretations are an important contribution to the growing literature on forecast reconciliation. The	authors do a very good job explaining the geometric aspects, which lead to new insights. That being said, the contribution of the paper in its current form is mainly theoretical, as the empirical evaluation is not much of a contribution.  I do not think the paper meets the high standard set by the IJF in terms of empirical evaluation. I recommend that the authors revise their paper with a particular focus on strengthening its empirical contribution to more clearly show the practical value of the geometric insights they derive. I hope that the authors will find my comments useful for improving their paper.}\\
		
	We thank the referee for their comments acknowledging the theoretical contributions of the paper.  While we believe that more theory focused papers lie within the scope of the International Journal of Forecasting, we also agree with the referee that the empirical section could be greatly improved.  We have endeavoured to do this in ways outlined in our responses below.\\
	
	\section*{Detailed Response to Referee 1: Major comments}
	
	
	\begin{enumerate}
		
		
		
		\item \textit{When the empirical evaluation is focused on bias, then it would make sense to include an error that measures bias in addition to the squared error.}\\
		
		We now also report the mean error for all methods to highlight bias. TO DO (GEORGE)\\
		
		\textit{Moreover, simply showing the MSE without any confidence intervals or measures of significance is not sufficient for the reader to assess the results.  It would also be useful to show the MSE relative to the base forecasts or the percentage improvement that is obtained.} \\
		
		We now include a Nemenyi comparison of the methods and report results in terms of percentage improvement over base forecasts. TO DO (GEORGE)\\
	
	    \textit{The best performing reconciliation method is MinT with shrinkage, but the authors never state	the value of the shrinkage parameter or how it was chosen.  Similarly, they compare with variance scaling without explaining what they mean by variance scaling.}\\
	    
	    The shrinkage estimator is that of Sch\"afer and Strimmer~(2005). Details on this method (including the choice of shrinkage parameter) are now provided.  Similarly, we now provide more detail on variance scaling, which simply refers to using the variance of one step ahead in-sample forecast errors.\\
	    
	    \textit{In addition to the above mentioned shortcomings, I think the authors should reconsider their empirical evaluation. Maybe a second case study or a simulation study is needed to show the value of the geometric insights provided. We already know that MinT is better than OLS and WLS. What is the new and better reconciliation approach that has come from the geometric insights?}\\	
		
		Propose a generalised MinT. TO DO (TAS).  Demonstrating these additional contributions in the context of our existing empirical example has lengthened the paper considerably.  Therefore, we prefer to refrain from including a second empirical demonstration.\\
		
		\item \textit{Improvement guarantees: The boxplot in Figure 8 shows that OLS
		always improves MSE, while this is not the case for the other reconcili-
		ation approaches. To gain a better understanding of the implications of
		Theorem 3.2, it would be useful to show that the other approaches always
		improve accuracy in their transformed spaces. What is the interpreta-
		tion of the transformed spaces and can the authors make the connection
		between these spaces and the choice of reconciliation approach and error
		measure more clear?}\\ 
	
	    We now report results for an MSE in the transformed space and show blah blah blah TO DO.
		
		\textit{For example, Hyndman et al. (2011); van Erven and
		Cugliari (2015) argued for selecting OLS to increase the importance of
		forecasting the aggregate. What is the argument for WLS or MinT and
		what is the corresponding consistent error measure?}\\
	    
	    We would like to point out that any argument made by Hyndman et al. (2011) or van Erven and Cugliari (2015) that OLS reconciliation somehow targets and improves the aggregate series is based on empirical rather than theoretical evidence.  Address this misconception, which appears to be common, has in fact been a major motivation behind us writing the paper.  We hope that with the revision to the paper these distinctions are clearer.\\
	    
	    \end{enumerate}
	    
	    \section*{Detailed Response to Referee 1:Minor Comments}
	    
	    \begin{enumerate}
	    
	    \item \textit{In the first half of the paper it feels like every other sentence includes a however. I suggest reducing the use of however.}\\
	    
	    We have reduced the usage of the word however substantially, from over 12 instances to just 3.\\
	    
	    \item \textit{P. 2, l. 11: In several places the authors talk about adjusting forecasts ex-post. Although I understand what is meant, it gives the impression that forecasts are adjusted after observing the realized values.}\\
	    
	    We have either removed all use of the term `ex post' or stated `ex post of base forecasting' to avoid the potential for confusion.\\
	    
	    \item \textit{P. 2, l. 12: The authors discuss the regression formulation of forecast reconciliation. It would be useful to also make the connection to the optimization formulation considered by, e.g., van Erven and Cugliari (2015); Nystrup et al. (2020). This could also be useful for clarifying the connection between reconciliation approaches and error measures.}\\
	    
	    Must reread these papers and comment.  TO DO (TAS)\\
	    
	    \item \textit{P. 4, l. 22: forf}\\
	    This has been corrected.\\
	    
	    \item \textit{P. 10, l. 12: the comma should not be there.}\\
	    The comma has been removed.\\
	    
	    \item \textit{P. 11, Figure 3: usually a small square is drawn in the corner of the triangle to show orthogonality.}\\
	    We have made this change to all Figures. TO DO (George)\\
	    
	    \item \textit{P. 17, l. 18: i,e.}\\
	    We believe that you are asking that we include a comma after this instance of `i.e' and we have made this correction.\\
	    
	    \item \textit{P. 26, l. 14: the authors mention that the full results are available upon request. I suggest including them in an online supplementary appendix.}\\
	    We now include these an an online supplement. TO DO\\
	    
	    \item \textit{P. 27, Conclusions: The authors should comment on the implications of the non-uniqueness of the S matrix for future work on cross-temporal reconciliation (Kourentzes and Athanasopoulos, 2019).}\\
	    TO DO (GEORGE).\\
	    
    \end{enumerate}

    \section*{Detailed Response to Referee 2}

	\begin{enumerate}	
	
	\item \textit{This type of paper makes me regret not investing more time into geometric interpretation because as shown here, it offers an elegant and intuitive way to showcase results related to data integration and reconciliation. The paper is extremely well written, with a great flow and thoughtful considerations. Figure 4 and its description are exemplary successful in their simplicity and effectiveness. The discussion of theorem 3.1 on page 11 is another example of thoroughness and clever insight.}\\
	
	We thank the referee for these kind comments.\\
	
    \item\textit{I found only one statement in the paper that could be better supported by evidence on page 8 lines 14 when the author(s) refer to multivariate modeling. State-space approaches have also been shown to be theoretically successful is solving these problems although maybe not to the large scale needed for very detailed and complex hierarchical systems. A comment or comparative discussion to the multivariate modeling may be useful.}\\
    
    We have rewritten this sentence and now explicitly include the case of state space models in our discussion.\\
    
    
    \item\textit{In the context of real-life application and either for small discussion here or future work, I am wondering if and how the author(s) coherent subspace that would be defined with hard boundaries. For example, the set of Australian tourism flow data and any forecasts that would be considered useful should be non-negative and likely upper bounded (if only by the size of the global population or other more realistic subject matter expert opinion). In many other reconciliation problems, these boundary constraints affect the feasible space. In the context here, could a convoluted case lead to an orthogonal projection be coherent but outside the desired constrained subspace?}\\
    It is indeed possible for an orthogonal projection to reconcile a set of positive base forecasts into a vector that includes some negative values.  This is a fairly pathological case and does not arise in the application that we consider (and many other applications that we have considered in other work).  We do agree that this problem may arise in other contexts so we now include some discussion of this issue and cite some recent work that addresses this problem. TO DO (TAS)\\

    \item\textit{Please correct the minor typo just before section 2 (forf).}\\
    We have made this correction.\\
    
    \end{enumerate}

\end{document}